\documentclass[twocolumns]{IEEEtran}
\usepackage{tikz}
\tikzstyle{box}=[rectangle,draw=black, ultra thick, minimum size=1cm]
\usepackage{algorithm}
\usepackage{algpseudocode}
\usepackage{graphicx}
\usepackage{caption}


\author{Alp Gokcek, Erdal Sidal Dogan, Mert Komurcuoglu\\\{gokcekal, doganer, komurcuoglum\}@mef.edu.tr\\ MEF University \\ \today}
\title{Rock-Paper-Scissors AI Agent}


\begin{document}
	\maketitle
\section{Introduction}

Rock-Paper-Scissors is a game that has been around for while and well known to almost everyone. The game is played with two players, goal is to defeat the opponent by making a choice that prevails the opponents. Rules to defeat is as follows;

\begin{enumerate}
	\item Paper\textgreater Rock
	\item Rock\textgreater Scissors	
	\item Scissors\textgreater Paper
\end{enumerate}

In the beginning of the game, two players make a random choice among the Rock-Paper-Scissor triplet simultaneously. Whoever made the choice that defeats the opponents choice wins the game.

Even though it seems that the choices are random and probabilities of winning for each player is equal, scientist discover that "if a player wins for a round, they are much more likely to win to following one too". Furthermore, the choices 





\end{document}

\bibliographystyle{IEEEtran}
\bibliography{ref.bib}